\begin{slide}{Motivace}
\begin{itemize} 
    \item proč programovat v Node.js a prát se s asynchronním kódem, když na webovém backendu potřebujeme asynchronnost jen čas od času?
    \item typicky potřebujeme vykonat nějakou úlohu, která trvá moc dlouho na to, aby mohla být vykonána v rámci HTTP requestu
    \item klientovi proto vrátíme odpověď s URL na kterou se může periodicky dotazovat nebo mu odpověď pošleme push notifikaci skrz web sockety
    \item samotná asynchronnost často nestačí, chceme např. jednoduše plánovat úlohu v závislosti na dokončení jiné úlohy
    \item to vše můžeme vyřešit synchronním kódem  pomocí asynchronní fronty
\end{itemize}
\end{slide}

\begin{slide}{Modelové případy}
\begin{itemize}
    \item chceme poslat odpověď klientovi bez čekání (dotaz na externí službu, náročný výpočet, logování, ...)
    \item opakování úkolů -- chceme notifikovat e-shop o stavu objednávky, ale ten nemusí být zrovna dostupný, proto to musíme zkusit znovu za 5 minut
    \item chceme paralelizovat výpočet, ale naše webová aplikace běží jen v jednom procesu
    \item nechceme zatěžovat webový server náročnými operacemi a radši je pošleme na jiný stroj
    \item máme sadu úkolů s různými typy závislostí
\end{itemize}
\end{slide}

\begin{slide}{Co je to Celery?}
\begin{itemize}
    \item asynchronní fronta napsaná v Pythonu
    \item existují bindingy pro jiné jazyky
    \item pro distribuci úkolů lze použít různá úložiště (RabbitMQ, Redis, relační databázi\dots)
    \item úkoly můžeme posílat do různých front na různých serverech
    \item každá fronta může být obsluhována libovolným počtem vláken
\end{itemize}
\end{slide}

\begin{slide}{Co je to Celery?}
TODO: obrazek front a workeru
\end{slide} 

\section{Reálný příklad} 
\begin{slide}{Popis problém}
 Vyvíjíme online platební metodu, kde objednávka musí projít netrivialním scoringovým systémem. Problémy se synchronním kódem: 
\begin{itemize}
    \item Scoring může trvat až 1 minutu.
    \item Scoring je výpočetně náročný a nesmí zatěžovat obsluhu jiných requestů.
    \item Pro zvýšení propustnosti aplikace chceme scoring různých objednávek paralelizovat.      
    \item Pro jednoho uživatele nesmí scoring běžet paralelně.
\end{itemize}
\end{slide}

\begin{slide}{Řešení pomocí Celery}
    
\end{slide}

\begin{slide}{Plánování úkolů a závislosti}
\end{slide} 
