\documentclass[xcolor=x11names,compress,t]{beamer}

%% General document %%%%%%%%%%%%%%%%%%%%%%%%%%%%%%%%%%
\usepackage{graphicx}
\usepackage[compatibility=false]{caption}
\usepackage{subcaption}
\usepackage[utf8x]{inputenc}
\usepackage{wrapfig}
\usepackage{hyperref}
\usepackage{minted}
\hypersetup{
  pdfauthor={Oskar Hollmann},
  pdftitle={Travis CI},
  breaklinks,
  colorlinks=true,
   linkcolor=black,
}

%% Beamer Layout %%%%%%%%%%%%%%%%%%%%%%%%%%%%%%%%%%
\useoutertheme[subsection=false,shadow]{miniframes}
\setbeamertemplate{navigation symbols}{}%remove navigation symbols

    \defbeamertemplate{footline}{author and page number}{%
      \usebeamercolor[fg]{page number in head/foot}%
      \usebeamerfont{page number in head/foot}%
      \hspace{1em}\insertshortauthor\hfill%
      \insertpagenumber\,/\,\insertpresentationendpage\kern1em\vskip2pt%
    }
    \setbeamertemplate{footline}[author and page number]{}

\useinnertheme{default}
\usefonttheme{serif}
\usepackage{kpfonts} % originally palatino font was used
\usepackage{verbatim}
\renewcommand*\ttdefault{cmtt}

\setlength{\leftmargini}{0pt} % reduce itemize margin
\setlength{\leftmargin}{0pt} % reduce standard text margin

% Use small-caps in appropriate places
\setbeamerfont{title like}{shape=\scshape}
\setbeamerfont{frametitle}{shape=\scshape}
\setbeamerfont{section in head/foot}{shape=\scshape}

% Colour settings
\setbeamercolor*{lower separation line head}{bg=DeepSkyBlue4} 
\setbeamercolor*{normal text}{fg=black,bg=white} 
\setbeamercolor*{alerted text}{fg=red} 
\setbeamercolor*{example text}{fg=black} 
\setbeamercolor*{structure}{fg=black} 
\setbeamercolor*{palette tertiary}{fg=black,bg=black!10} 
\setbeamercolor*{palette quaternary}{fg=black,bg=black!10} 

% Fix incorrect alignment of itemize in right columns
\makeatletter
\define@key{beamerframe}{t}[true]{% top
  \beamer@frametopskip=0cm plus 0\paperheight\relax%
  \beamer@framebottomskip=0pt plus 1fill\relax%
  \beamer@frametopskipautobreak=\beamer@frametopskip\relax%
  \beamer@framebottomskipautobreak=\beamer@framebottomskip\relax%
  \def\beamer@initfirstlineunskip{}%
}
\makeatother

% Shortcuts for the column environment
\renewcommand{\(}{\begin{columns}[T]}
\renewcommand{\)}{\end{columns}}
\newcommand{\<}[1]{\begin{column}{#1}}
\renewcommand{\>}{\end{column}}

\newenvironment{slide}[1]{\subsection{#1} \begin{frame}{#1}}{\end{frame}}

%%%%%%%%%%%%%%%%%%%%%%%%%%%%%%%%%%%%%%%%%%%%%%%%%%

\title{Travis CI\\\vspace{0.3em}\includegraphics[width=80pt]{./travis-logo.png}}
\subtitle{Continuous Integration}
\author[Oskar Hollmann]{Oskar Hollmann}
\institute{User Technologies}
\date{26.5.2016}

\begin{document}

\maketitle
\setbeamertemplate{logo}{\includegraphics[height=0.8cm]{./travis-logo.png}}

\section{Motivation}

\begin{slide}{Have you met Mr. Travis?}
\begin{itemize}
	\item Travis is a Continuous Integration platform,
    \item free for open-source projects,
    \item developed in Berlin, Germany since 2011,
    \item used by prominent projects such as Node.js or Ruby on Rails.
\end{itemize}
\begin{figure}[htb]
    \centering
    \includegraphics[width=0.7\textwidth]{ci}
    \caption{Continuous Integration diagram [\href{http://www.360logica.com/blog/2014/06/continuous-integration.html}{360logica.com}].}
\end{figure}
\end{slide}

\begin{slide}{What can Travis do for you?}
\begin{enumerate}
	\item Automatically build Git repositories or pull requests.
    \item Run all tests.
    \item Integrate code coverage tools and code quality checks.
    \item Deploy the code on AWS, Heroku, PyPi, RubyGems, \dots
\end{enumerate}
\begin{figure}[htb]
    \centering
    \includegraphics[width=0.7\textwidth]{ci-travis-parts}
    \caption{Continuous Integration diagram}
\end{figure}
\end{slide}

\section{Using Travis}
\begin{slide}{1. Detection of pull request}
\begin{figure}[htb]
    \centering
    \includegraphics[width=\textwidth]{detection-github}
    \caption{Travis CI integrates nicely with GitHub.}
\end{figure}
\end{slide}

\begin{slide}{2. Test runs}
\begin{figure}[htb]
    \centering
    \includegraphics[width=\textwidth]{detection-travis}
    \caption{Tests are run using chosen versions of languages and libraries.}
\end{figure}
\end{slide}

\begin{slide}{2. Test runs}
\begin{figure}[htb]
    \centering
    \includegraphics[width=\textwidth]{success-travis}
\end{figure}
\end{slide}

\begin{slide}{3. Report results}
\begin{figure}[htb]
    \centering
    \includegraphics[width=\textwidth]{success-github}
\end{figure}
\vspace{-0.5em}
\begin{figure}[htb]
    \centering
    \includegraphics[width=0.8\textwidth]{success-email}
    \caption{Results are pushed to Github and e-mail.}
\end{figure}
\end{slide}

\section{Configuration}

\subsection{}
\begin{frame}[fragile]{Configuration}
\begin{itemize}
    \item Tests run in a sandboxed Ubuntu Docker image.
    \item Configuration contained in a single \textbf{.travis.yml} file.
    \item Many stages of the test ran can be customised:
    \begin{itemize}
      \item \verb+addons+
      \item \verb+before_install+
      \item \verb+install+
      \item \verb+before_script+
      \item \verb+script+
      \item \verb+after_success/after_failure+
      \item \verb+before_deploy/deploy/after_deploy+
      \item \verb+after_script+
    \end{itemize}
\end{itemize}
\begin{minted}[frame=single]{yaml}
addons:
  apt:
    packages:
    - cmake
\end{minted}
\end{frame}

\subsection{}
\begin{frame}[fragile]{Example Configuration File}
\begin{minted}[frame=single]{yaml}
language: python

# Which version of the language to support?
python:
  - "2.7"
  - "3.5"

# Enviroment variables can be used to setup builds
# and test multiple versions of libraries
env:
  - DJANGO_VERSION=1.7
  - DJANGO_VERSION=1.8
  - DJANGO_VERSION=1.9
\end{minted}
\end{frame}

\subsection{}
\begin{frame}[fragile]{Example Configuration File}
\begin{minted}[frame=single]{yaml}
# All combinations of env. vars are tested
matrix:
  exclude: # But some may not be supported
    - python: "3.3"
      env: DJANGO_VERSION=1.9

# Install required dependencies
install:
  - cd example
  - pip install -r requirements/base.txt
  - pip uninstall -y -q django
  - pip install -q Django==$DJANGO_VERSION

# We can override the default build step
script:
   coverage run manage.py test app.tests
\end{minted}
\end{frame}

\begin{slide}{}
\vspace{5em}
\centering
\Large
Thank you for your attention.

\vspace{5em}

Any questions?
\end{slide}
\end{document}
