    \documentclass[xcolor=x11names,compress,t]{beamer}

%% General document %%%%%%%%%%%%%%%%%%%%%%%%%%%%%%%%%%
\usepackage{graphicx}
\usepackage[compatibility=false]{caption}
\usepackage{subcaption}
\usepackage[utf8x]{inputenc}
\usepackage{wrapfig}
\usepackage{hyperref}
\usepackage[linenos]{minted}
\hypersetup{
  pdfauthor={Oskar Hollmann},
  pdftitle={Celery},
  breaklinks,
  colorlinks=true,
  linkcolor=DeepSkyBlue4,
  urlcolor=DeepSkyBlue4,
}

%% Beamer Layout %%%%%%%%%%%%%%%%%%%%%%%%%%%%%%%%%%
\useoutertheme[subsection=false,shadow]{miniframes}
\setbeamertemplate{navigation symbols}{}%remove navigation symbols

    \defbeamertemplate{footline}{author and page number}{%
      \usebeamercolor[fg]{page number in head/foot}%
      \usebeamerfont{page number in head/foot}%
      \hspace{1em}\insertshortauthor\hfill%
      \insertpagenumber\,/\,\insertpresentationendpage\kern1em\vskip2pt%
    }
    \setbeamertemplate{footline}[author and page number]{}

\useinnertheme{default}
\usefonttheme{serif}
\usepackage{kpfonts} % originally palatino font was used
\usepackage{verbatim}
\renewcommand*\ttdefault{cmtt}

\setlength{\leftmargini}{0pt} % reduce itemize margin
\setlength{\leftmargin}{0pt} % reduce standard text margin

% Use small-caps in appropriate places
\setbeamerfont{title like}{shape=\scshape}
\setbeamerfont{frametitle}{shape=\scshape}
\setbeamerfont{section in head/foot}{shape=\scshape}

% Colour settings
\setbeamercolor*{lower separation line head}{bg=DeepSkyBlue4} 
\setbeamercolor*{normal text}{fg=black,bg=white} 
\setbeamercolor*{alerted text}{fg=red} 
\setbeamercolor*{example text}{fg=black} 
\setbeamercolor*{structure}{fg=black} 
\setbeamercolor*{palette tertiary}{fg=black,bg=black!10} 
\setbeamercolor*{palette quaternary}{fg=black,bg=black!10} 

% Fix incorrect alignment of itemize in right columns
\makeatletter
\define@key{beamerframe}{t}[true]{% top
  \beamer@frametopskip=0cm plus 0\paperheight\relax%
  \beamer@framebottomskip=0pt plus 1fill\relax%
  \beamer@frametopskipautobreak=\beamer@frametopskip\relax%
  \beamer@framebottomskipautobreak=\beamer@framebottomskip\relax%
  \def\beamer@initfirstlineunskip{}%
}
\makeatother

% Shortcuts for the column environment
\renewcommand{\(}{\begin{columns}[T]}
\renewcommand{\)}{\end{columns}}
\newcommand{\<}[1]{\begin{column}{#1}}
\renewcommand{\>}{\end{column}}

\newenvironment{slide}[1]{\subsection{#1} \begin{frame}{#1}}{\end{frame}}
\newmintedfile[pythonfile]{python}{linenos}
%%%%%%%%%%%%%%%%%%%%%%%%%%%%%%%%%%%%%%%%%%%%%%%%%%

\title{Celery}
\subtitle{A Distributed Task Queue}
\author[Oskar Hollmann]{\vspace{0.3em}\includegraphics[width=0.2\textwidth]{celery-logo}\\Oskar Hollmann}
\institute{User Technologies}
\date{26.10.2016}

\begin{document}

\maketitle

\section{Motivation}

\begin{slide}{Motivation}
    \begin{itemize} 
        \item time demanding tasks are a pain in web apps
            \begin{itemize}
                \item HTTP request can easily time out
                \item it's not acceptable to block the client for too long
                \item client may not care about the result
            \end{itemize}
        \item either we return an URL to the client who polls it later to get the result
        \item or we push it through web sockets
        \item how can we achieve that?
    \end{itemize}
\end{slide}

\begin{slide}{Motivation}
    \vspace{5em}
    \centering 
    {\Huge Node.js to the rescue!}
    \includegraphics[height=9em]{nodejs-logo}
\end{slide}

\begin{slide}{Motivation}
    \vspace{1em}
    \centering 
    {\large \emph{Node.js is cool, but...}}
    \vspace{1em}
    \begin{itemize} 
        \item Why use Node.js and struggle with async code when async operations are seldom needed in a web app?
        \item Async code is not enough -- we might need to distribute the tasks, run them periodically, \dots
        \item Node.js programmers are a rare commodity.
        \item What we actually need is \textbf{an asynchronous task queue}.
        \item Examples: RabbitMQ, JMS, Celery, \dots
    \end{itemize}

    \vspace{1em}
    \includegraphics[height=2em]{rabbitmq-logo}\hspace{2em}\includegraphics[height=2em]{celery-logo-long}
\end{slide}


\begin{slide}{Use Cases of Distributed Task Queues}
    \begin{enumerate}
        \item Non-blocking task execution.
        \item Task execution with failure recovery.
        \item Concurrent task execution for single-threaded apps.
        \item Distribute task to other machines.
        \item Handle complex task workflows with dependencies.
        \item Periodic tasks.
    \end{enumerate}
\end{slide}


\section{About Celery}
\begin{slide}{Task Queue in Web App}
    \begin{figure}
        \includegraphics[scale=0.42]{django-celery-architecture}
        \caption{Source \href{http://en.proft.me/2013/10/25/celery-periodic-tasks-django-projects/}{en.proft.me}}
    \end{figure}
\end{slide}

\begin{slide}{Celery is}
    \begin{itemize}
        \item distributed task queue written in Python
        \item bindings for: PHP, Ruby, NodeJS and more
        \item different message broker transports: Redis, RabbitMQ, MongoDB and more
        \item arbitrary number of queues and workers
    \end{itemize}
\end{slide}

\begin{slide}{Workers and Queues}
    \begin{figure}
        \includegraphics[scale=0.45]{celery-queues-and-workers}
        \caption{Source \href{https://abhishek-tiwari.com/post/amqp-rabbitmq-and-celery-a-visual-guide-for-dummies}{abhishek-tiwari.com}}
    \end{figure}
\end{slide} 

\subsection{}
\begin{frame}[fragile]{Minimal Example -- Create Celery App}
\pythonfile{celery_example.py}
\end{frame}

\subsection{}
\begin{frame}[fragile]{Minimal Example -- Calling Tasks}
\begin{minted}[linenos]{python}
from celery_example import add

# Fire up the task
add.delay(1, 4)

# Or a more sophisticated way
res = add.apply_async(args=(2, 4), queue='my_queue')

res.status  # Get status of the task
res.get()  # Wait for the result
\end{minted}
\end{frame}

\section{Task Orchestration}
\subsection{}
\begin{frame}[fragile]{Task Orchestration with Signatures}
\begin{itemize}
    \item Celery supports complex workflows using different orchestration primitives
    \item \textbf{group, chain, chord, map, starmap, and chunks}
    \item for these, we need to be able to create a task instance with the required parameters but not execute it
\end{itemize}
\begin{minted}[linenos]{python}
from celery import signature

# Signature is a task with fixed parameters that 
# can be called later (similar to partial in FP)
signature('tasks.add', args=(2, 2))

# Signature shortcut
s = add.s(2, 2)
s.delay()
\end{minted}
\end{frame} 

\subsection{}
\begin{frame}[fragile]{Supported Workflows}
\begin{minted}[linenos]{python}

@app.task
def xsum(numbers):
    return sum(numbers)

# 1. Chain passes result of the task to the next one.
(add.s(2, 2) | add.s(4) | add.s(8))().get() # → 16

# 2. Group executes a set of tasks in paralel.
group(add.s(i, i) for i in range(4))() # → [0, 2, 4, 6]

# 3. Chord is a group that takes a callback.
chord((add.s(i, i) for i in range(4)), xsum.s())() # → 12

# 4. Map applies task to a list of args in 1 message.
xsum.map([1, 2, 3], [7, 10]]) # → [6, 17]

\end{minted}
\end{frame} 

\subsection{}
\begin{frame}[fragile]{Example of a Complex Workflow}
\begin{minted}[linenos]{python}

@app.task
def create_user(username, first, last, email):
    ...
@app.task
def import_contacts(user):
    ...
@app.task
def send_welcome_email(user):
    ...

new_user_workflow = (
    create_user.s() | group(import_contacts.s(),
                            send_welcome_email.s()))

new_user_workflow.delay('asterix', 'Asterix', 'Gaul', 
                        'asterix@rychmat.eu')
\end{minted}
\end{frame} 

\section{Case Study} 
\begin{slide}{Problem Specification}
    \begin{itemize}
        \item We have an online payment method where each order must go through non-trivial scoring process.
        \item Problems with synchronous code:
        \begin{enumerate}
            \item Scoring may take up tu a minute.
            \item The computation is resource-heavy and must not affect processing of new orders.
            \item To increase throughput of the app, different scoring tasks must be run concurrently.
            \item Scoring cannot run in parallel for one customer.
        \end{enumerate}
    \end{itemize}
\end{slide}


\subsection{}
\begin{frame}[fragile]{Implementation in Celery}
\begin{minted}[linenos]{python}
# Setup Celery in Django settings
BROKER_URL = 'redis://localhost:6379/0'
BROKER_TRANSPORT = 'redis'
CELERY_SCORING_WORKERS = 10
CELERY_QUEUES = [
    Queue(str(i), Exchange(str(i)), routing_key=str(i)) 
    for i in range(CELERY_SCORING_WORKERS)]

# Define the scoring task
@app.task(name='run_scoring', time_limit=3600)
def execute_scoring(order):
    ...
# Run scoring when new order arrives
execute_scoring.apply_async(
    args=(order,),
    queue=str(customer.pk % CELERY_WORKERS_COUNT))
\end{minted}
\end{frame}

\begin{slide}{}
\begin{figure}
    \includegraphics[height=0.7\textheight]{porkey}
\end{figure}

\vspace{-1em}
\centering
\Large{Thanks for your attention!}
\vspace{1em}

\footnotesize{Code examples adapted from \href{http://celery.readthedocs.io}{Celery docs}.}
\end{slide}
\end{document}
